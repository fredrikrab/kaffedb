\section{Brukerhistorier}

\subsection{Brukerhistorie 1}

\begin{tabular}{ | p{0.95\textwidth} }
    \textsl{En bruker smaker kaffen Vinterkaffe 2022 fra Trondheims-brenneriet Jacobsen \& Svart (brent 20.01.2022), gir den 10 poeng og skriver «Wow – en odyssé for smaksløkene: sitrusskall, melkesjokolade, aprikos!».
    Kaffen er lysbrent, bærtørket Bourbon (c. arabica), kommer fra gården Nombre de Dios (1500 moh.) i Santa Ana, El Salvador, har en kilopris på 600 kr og er ifølge brenneriet «En velsmakende og kompleks kaffe for mørketiden».
    Kaffen ble høstet i 2021 og gården fikk utbetalt 8 USD per kg kaffe.
    Input fra brukerens side er brenneri, kaffenavn, poeng og smaksnotat.}
\end{tabular} \\

Gitt opplysningene har vi følgende informasjon i \class{coffee}-tabellen:

\sqltable{coffee} \\

Brukerens input skal skrives til \class{reviews}-tabellen som har disse kolonnene:

\sqlcolumns{reviews}


\verb|user_email| tilsvarer brukerens epost-adresse (primærnøkkel i \class{users}-tabellen).
Vi antar at dette kan hentes fra innsendingsskjemaet (f.eks. fordi brukeren er logget på systemet).
Ettersom smakstidspunkt ikke er oppgitt, settes \verb|date_time| til nåtid av databasesystemet. Poengsum og smaksnotat hentes fra input.

Da gjenstår det å hente primærnøkkelen til "Vinterkaffe 2022" fra tabellen for kaffetyper (\textbf{coffee.}\verb|coffee_id|), slik at \textbf{reviews.}\verb|coffee_id| peker på denne kaffetypen.

Databasesystemet vårt krever at kombinasjonen (kaffenavn, brenneri) er unik,
altså at det er maks ett tilfelle av en kombinasjon (\verb|coffee_name|, \verb|roastery_name|) i \class{coffee}-tabellen.

Følgende spørring vil derfor gi ønsket \verb|coffee_id|: \\

\begin{lstlisting}[language=SQL]
    SELECT coffee_id FROM coffee
    WHERE roastery_name = 'Jacobsen & Svart' AND coffee_name = 'Vinterkaffe 2022';
\end{lstlisting}

Innsetting i database kan gjøres med: \\

\begin{lstlisting}[language=SQL]
    INSERT INTO reviews (review_id, user_email, coffee_id, date_time, rating, note)
    VALUES (
        null, 'epost@server.no',
        (SELECT coffee_id FROM coffee WHERE coffee_name == 'Vinterkaffe 2022'),
        datetime('now'), 10,
        'Wow - en odyssé for smaksløkene: sitrusskall, melkesjokolade, aprikos!'
    );
\end{lstlisting}

\subsection{Brukerhistorie 2}

\begin{tabular}{ | p{0.95\textwidth} }
    \textsl{En bruker skal kunne få skrevet ut en liste over hvilke brukere som har smakt flest unike kaffer så langt i år, sortert synkende.
    Listen skal inneholde brukernes fulle navn og antallet kaffer de har smakt.}
\end{tabular} \\

Vi kan finne antall unike kaffer som brukere har smakt ved å telle rader i \class{reviews}-tabellen med unik (\verb|user_email|, \verb|coffee_id|).
Tuppelen må være unik siden en bruker kan ha smakt samme kaffe mange ganger.
Vi kan avgrense tellingen til årets smaker ved å kreve at \verb|date_time| skal være større enn eller lik begynnelsen av året.

Fullt navn hentes fra \class{users}-tabellen via en \code{JOIN}-operasjon.
Resultatet grupperes etter \verb|user_email| for å skille mellom eventuelle brukere med samme navn. \\

\begin{lstlisting}[language=SQL]
    SELECT name, count(*)
    FROM (  SELECT DISTINCT user_email, coffee_id FROM reviews
            WHERE date_time >= datetime('now', 'start of year') )
    JOIN users USING(user_email)
    GROUP BY user_email
    ORDER BY count(*) DESC;
    
\end{lstlisting}

\subsection{Brukerhistorie 3}

\begin{tabular}{ | p{0.95\textwidth} }
    \textsl{En skal kunne se hvilke kaffer som gir forbrukeren mest for pengene ifølge KaffeDBs brukere (høyeste gjennomsnittsscore kontra pris), sortert synkende.
    Listen skal inneholde brennerinavn, kaffenavn, pris og gjennomsnittsscore for hver kaffe.}
\end{tabular} \\

Gjennomsnittsscore beregnes ved å hente rating-verdier fra \class{reviews}-tabellen og gruppere etter kaffetype.
Resten av opplysningene ligger i \class{coffee}-tabellen som kan skjøtes på med \code{JOIN}.
Sorteringsnøkkelen kan være gjennomsnittsscore delt på pris.

En spørring kan derfor se slik ut: \\

\begin{lstlisting}[language=SQL]
    SELECT roastery_name, coffee_name, kilo_price_nok, AVG(rating)
    FROM (SELECT rating, coffee_id FROM reviews)
    JOIN coffee USING(coffee_id)
    GROUP BY coffee_id
    ORDER BY AVG(rating)/kilo_price_nok DESC;
\end{lstlisting}

\subsection{Brukerhistorie 4}

\begin{tabular}{ | p{0.95\textwidth} }
    \textsl{En bruker søker etter kaffer som er blitt beskrevet med ordet «floral», enten av brukere eller brennerier.
    Brukeren skal få tilbake en liste med brennerinavn og kaffenavn.}
\end{tabular} \\

Tekstsøk kan gjøres med \code{LIKE '\%floral\%'} i \textbf{reviews.}\verb|note| og \textbf{coffee.}\verb|coffee_description|.\footnote{ Se også \href{https://www.sqlite.org/fts5.html}{SQLite FTS5 Extension} som kan vurderes dersom det etter hvert stilles store krav til ytelse.}
Samlet liste kan deretter presenteres ved å ta \code{UNION} av resultatene. \\

\begin{lstlisting}[language=SQL]
    SELECT roastery_name, coffee_name
    FROM (  SELECT coffee_id FROM reviews
            WHERE note LIKE '%floral%'  )
    JOIN coffee USING(coffee_id)
    UNION
    SELECT roastery_name, coffee_name FROM coffee
    WHERE coffee_description LIKE '%floral%';
\end{lstlisting}

\subsection{Brukerhistorie 5}

\begin{tabular}{ | p{0.95\textwidth} }
    \textsl{En annen bruker er lei av å bli skuffet av vaskede kaffer og deres tidvis kjedelige smak, og ønsker derfor å søke etter kaffer fra Rwanda og Colombia som ikke er vaskede.
    Systemet returnerer en liste over brennerinavn og kaffenavn.}
\end{tabular} \\

Brukeren ønsker å filtrere kaffetyper på henholdsvis land (\textbf{farms.}\verb|farm_country|) og foredlingsmetode (\textbf{refinement\_method.}\verb|refinement_name|).
Denne filtreringen kan i første omgang gjøres med \code{SELECT}, \code{FROM} og \code{WHERE} i de respektive tabellene.

Alle kaffetyper er fremstilt fra et bestemt parti med kaffebønner, som i vår database er representert av tabellen \class{batches}: \\

\sqlcolumns{batches}{}

Siden både gård (\verb|farm_name|) og foredlingsmetode (\verb|refinement_name|) er fremmednøkler i denne tabellen,
vil en \code{NATURAL JOIN} med tidligere filtrering gi alle parti som matcher brukerens kriterier. 

En videre \code{NATURAL JOIN} med \verb|coffee|-tabellen gir brennerinavn og kaffenavn. \\

\begin{lstlisting}[language=SQL]
    SELECT roastery_name, coffee_name
    FROM (
            SELECT farm_name FROM farms
            WHERE farm_country IN ('Rwanda', 'Colombia')
        ), (
            SELECT refinement_name FROM refinement_methods
            WHERE refinement_name != 'Vasket'
        )
    NATURAL JOIN batches
    NATURAL JOIN coffee;
\end{lstlisting}
