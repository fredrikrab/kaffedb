\documentclass[12pt]{article}
\usepackage{graphicx} % Allows you to insert figures
\usepackage{amsmath} % Allows you to do equations
\usepackage{fancyhdr} % Formats the header
\usepackage{geometry} % Formats the paper size, orientation, and margins
\usepackage[normalem]{ulem} % [normalem] prevents the package from changing the default behavior of `\emph` to underline.
\usepackage{array} % For tables
\usepackage[table,svgnames]{xcolor} % for tables
\usepackage{enumitem} % Better lists
\usepackage{csquotes} % Better text quoting
\usepackage{parselines}
\linespread{1.25} % about 1.5 spacing in Word
\setlength{\parindent}{0pt} % no paragraph indents
\setlength{\parskip}{1em} % paragraphs separated by one line 
\graphicspath{ {./images/} } 

\usepackage[format=plain,
            font=it]{caption} % Italicizes figure captions
\usepackage[english]{babel}
\renewcommand{\headrulewidth}{0pt}
\geometry{a4paper, portrait, margin=1in}
\setlength{\headheight}{14.49998pt}

\setlist[itemize]{leftmargin=*} % Remove indent from lists

% Class formatting
\newcommand{\class}[1]{\textbf{\ttfamily{#1}}}

% URLs and linking
\usepackage{hyperref}
\hypersetup{
    colorlinks=true,
    linkcolor=blue,
    filecolor=magenta,      
    urlcolor=cyan
    }
\urlstyle{same}

% Footnote rule style
\renewcommand{\footnoterule}{%
  \kern -3pt
  {\color{darkgray} \hrule width 2in}
  \kern 5pt
}

\newcommand\titleofdoc{KaffeDB} %%%%% Put your document title in this argument
\newcommand\GroupName{Gruppe 170} %%%%% Put your group name here. If you are the only member of the group, just put your name

% SQL read db
\usepackage{fontspec}
\usepackage{luacode}
\usepackage{luapackageloader}

\begin{luacode*}
  -- initialize path searching for Debian libraries
  local my_path = "/usr/share/lua/5.3"
  local my_cpath ="/usr/lib/x86_64-linux-gnu/lua/5.3/"
  package.path  = package.path  .. string.format(";%s/?.lua;%s/?/init.lua", my_path, my_path)
  package.cpath = package.cpath .. string.format(";%s/?.so;%s/?/init.so", my_cpath, my_cpath)
\end{luacode*}

\directlua{%
sql_script = require("sql_script")%
}

\newcommand{\sqlcolumns}[2]{%
\begin{minipage}{\textwidth}
  \directlua{sql_script.print_columns("#1", "#2")}%
\end{minipage}
}

\newcommand{\sqltable}[1]{%
  \directlua{sql_script.print_table("#1")}%
}

\setlength{\tabcolsep}{10pt}
\renewcommand{\arraystretch}{1.4}
\newcolumntype{h}{>{\columncolor{light-gray}} l}
\newcolumntype{t}{>{\ttfamily\color{darkred}} l}

% Inline code
\definecolor{inline-code-gray}{gray}{0.92}
\newcommand{\code}[1]{\colorbox{inline-code-gray}{\lstinline|#1|}}

% SQL syntax highlighting
\usepackage{listings}
\usepackage{color}
\usepackage{framed}

\definecolor{brickred}{rgb}{0.8, 0.25, 0.33}
\definecolor{darkred}{rgb}{0.55, 0.0, 0.0}
\definecolor{dkgreen}{rgb}{0,0.6,0}
\definecolor{gray}{rgb}{0.5,0.5,0.5}
\definecolor{mauve}{rgb}{0.58,0,0.82}
\definecolor{light-gray}{gray}{0.95}
\definecolor{shadecolor}{gray}{0.95}

\lstset{language=SQL,
  backgroundcolor = \color{light-gray},
  basicstyle={\linespread{1.5}\footnotesize\ttfamily},
  belowskip=3mm,
  breakatwhitespace=true,
  breaklines=true,
  classoffset=0,
  columns=flexible,
  commentstyle=\color{dkgreen},
  escapechar=§,
  extendedchars=true,
  framexleftmargin=0.0em,
  frame=single,
  keywordstyle=\color{blue},
  literate=%
    {é}{{\'{e}}}1
    {ø}{\o{}}1,
  morekeywords={*,datetime,DATETIME,references,REFERENCES},
  numbers=none, %If you want line numbers, set `numbers=left`
  numberstyle=\tiny\color{gray},
  showstringspaces=false,
  stringstyle=\color{mauve},
  tabsize=3,
  xleftmargin=0.0em
}

\lstset{emph={%  
    datetime, references%
    },emphstyle=\color{blue}%
}%